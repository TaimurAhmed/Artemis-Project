%
% File: chap04.tex
% Author: ta16969
%
\let\textcircled=\pgftextcircled
\chapter{Conclusion}
\label{chap:Conclusion}


\initial{T}his research review has explored the theoretical and practical underpinnings upon which to build a TEL solution. Therefore Chapter 02 has discussed and concluded the functional and creative considerations for developing a web based social network as a TEL solution; whereas Chapter 03 has concluded the technical technological considerations and best practice to be adopted during the actual development of the proposed application.

Chapter 02 justified the reason for exploring pedagogy and learning methods when developing a TEL solution, before identifying the shortcoming of  traditional pedagogy and methods. On the basis of improved learning outcomes, a blended learning methodology has been identified as the best solution; With the concluding caveat that the proposed TEL solution will enable the best practice of allowing instructor interventions and personalised prompts.

The chapter then identified flexible pedagogy to embed and refine the design rationale; enabling blended learning to allow students greater control over their learning process. To do so the deployment of a web based social network has been proposed as compatible with the findings and proposed design criteria. Inspiration from a recent trend in TEL i.e. OERs are to be considered when developing the solution, to create an open source adaptive learning environment; a scalable VLE. A participatory learning model, adapted from the study of neural networks to enhance collaborative learning, was used to illustrate interaction within the social network and temper the adaptation/enrichment of the resultant VLE. Effectively Chapter 02 concludes the proposed rationale for the creative design of the TEL solution; A web based social network that enables an instructor lead blended methodology and flexible student learning via an adaptive VLE for collaborative learning, disseminated as an OER.

Chapter 03 progressed from the creative aspects of the TEL solution in the previous chapter, to the more technical relevant for development of the TEL solution; Citing subsets of captology i.e. persuasiveness of computer technology as a critical success factor for the TEL solution. The importance of this is reiterated based on initial meetings with the TEL department at the UoB which cited lack of user acceptance as a key reason for failure of certain perfectly functional TEL solutions adopted by the UoB. Therefore the chapter outlines best practice to be used during development; for improving the Web Credibility of a web solution during development and considering Web Application security best practice to use for to protect to users (as per the OWASP Top 10 recommendations for 2017), indirectly enhancing credibility. Finally best practice to accommodate user accessibility for the visually impaired via relevant technologies such as POSH HTML, semantic CSS \& JS and WAI-ARIA were identified and explained.

Chapter 04 explained the project milestones based work plan. It is believed that via the achievement of constituent tasks and the realisation of their resultant deliverables, the creative and technical design aspects of the envisioned TEL solution will be realised; To create the proposed solution. The deliverables were summarised in a list, work flow illustrated via a project planning tool, a risk identification \& mitigation strategy was developed and the workplan (including a minimum viable product with extensions) was explained.

Having explored the theoretical underpinning to inspire the creative design and illustrate the technical consideration for the TEL solution, it is believed that during the course of the project the focus will now shift predominantly towards the actual development and the practical application of theory and best practice. Increasing the depth of the theoretical underpinnings will be unlikely beyond further illustration and explanation. It is expected that the practical implications and application of best practice will be a more important feature of the project during the development stage. This seems reasonable as before the project is developed, it is beyond the reasonable constraints to try and predict every eventuality during the web development process and how it can be mitigated apart from accepting the proposed adoption of  best practice.