%
% File: chap01.tex
% Author: ta16969
% Description: Introduction.
%
\let\textcircled=\pgftextcircled
\chapter{Introduction}
\label{chap:intro}

\initial{T}his chapter serves as a semantic guide for the document. It introduces the project background by outlining the aims, objectives and overall structure of the document. This chapter will conclude with the scope  outlined; vide the key areas of the research project.


%=======
\section{Project Motivation}
\label{sec:sec01}


The inspiration behind this project stems from the observation of a real world problem; Currently, incoming Computer Science (CS) cohorts at the University of Bristol (UoB) commence studies before the term officially starts via \textit{week zero} (W-0). W-0 is effectively a week long foundation skills development course before the commencement of the first official teaching block (TB-1).

The underlying objective of W-0 is commendable; It satisfies the necessity of orienting a technically and culturally diverse cohort to the main programming languages and procedural norms of the CS department at the UoB (from here on referred to as the CS department). However despite solving the aforementioned practical necessity for effective course delivery, W-0 is not without it shortcomings. Some examples are listed as follows:

\begin{itemize}

    \item W-0 overlaps with the official university wide Induction Week (IW) for UoB students which inevitably results in scheduling clashes. It can be reasonably deduced that scheduling clashes and student failure to attend either event may compromise the underlying objectives of both IW and W-0.
    
    \item The CS department and the UoB both allocate resources towards W-0 and IW respectively. A deterioration in the underlying purpose of allocating these resources is arguably a wastage of limited teaching and administrative resources.
    
    \newpage
    
    \item W-0's traditional pedagogical delivery does not recognise the individual learning needs of a technically diverse cohort from various academic and professional backgrounds; Effectively, W-0 does not make allowances for students to identify and address their own learning needs, at a time, place or mode that is convenient to them or allows them to personalise their learning experience.
    
    An example of what this may imply is that a student with prior knowledge or experience in the subject matter may wish to dedicate their resources elsewhere until TB-1 officially begins. Conversely those with insufficient basic experience may not be given the opportunity to reasonably address their learning needs to get the best possible experience from this course. This can result in poor learning outcomes, demotivation and arguably even students dropping out of the course after having started it.
    
    \item Compromised learning and induction quality places undue costs on students; The very existence of IW and W-0 are indicative of a recognition of student needs that needs to be mitigated. This can cover a wide range of needs e.g. difficulties in adapting to a new environment and meeting learning objectives effectively. Compromising them is likely counterproductive and aggravates avoidable student costs that the UoB as a whole is seeking to avoid.
    
    \item Despite the subject matter of W-0 being relatively static; the same traditional teaching resources need to be dedicated to delivering it each year. This implies a process that can be optimised via a technological solution with either a non-recurrent or reduced allocation of resources. Alternatively an incrementally enriching TEL solution may be developed, that makes use of the recurrent allocation of resources i.e. improves over the long term with time and investment.
    
\end{itemize}




%==New Section
\section{Aims and Objectives}
\label{sec:sec01}

The primary aim of the project is to create a Technology Enhanced Learning (TEL) solution, to substitute/complement existing traditional teaching delivery methods utilised by the CS programmes at the UoB.

A resultant secondary aim is also proposed to satisfy the problems cited in the Project Motivation section earlier; Development of a TEL tool which may enable instructors to use modern pedagogical practice to improve learning outcomes and benefit students by giving them greater control over the learning process. It is also proposed that a TEL solution be designed with the view of making it scalable and malleable, via open source considerations.

\newpage
The aims of this project will be realised by achieving the following objectives:

\begin{enumerate}

    \item Analysing and reviewing relevant literature, covering a multi-cognate spread of broad topics (explained in the scope) for the purpose of creating a strong theoretical foundation upon which to inspire the creative and technical design of the TEL solution. 
    
    \item Identifying the stakeholders of a TEL solution for the purpose of understanding stakeholders and their values. An understanding of stakeholder requirements and criteria will help temper the findings of secondary research and inspire the development of a holistic TEL solution, that will be successful for its specific environment.
        
    \item Amalgamating the result of the aforementioned research and incorporating it into the creative and technical design of a TEL solution.
    
    \item Appropriately disseminating the code, developer notes and user manuals to make it open source so as to benefit scalability. In addition to this open source or collaborative user notes may also be created so as to enhance the efficacy of the TEL solution.
    
    \item Furthermore it is proposed that the project aims can be enhanced via the achievement of the following \textbf{optional objectives}:
    \begin{enumerate}
    \item Making technical and creative design considerations to benefit user accessibility and therefore the TEL solution's efficacy.
    \item Making technical design considerations to enhance applcation security and therefore make the technology more \textit{reliable} towards the benefit of users. As discussed later, enhancing the credibility or reliability of a TEL solution may improve its efficacy.
    \item Evaluating the TEL solution based on the creative and technical design considerations gained via research to design the TEL solution.
    \end{enumerate}
\end{enumerate}







%==New Section
\section{Deliverables}
\label{sec:sec01}

The \textbf{key deliverables} for this project are a TEL solution and a corresponding thesis. These key deliverables will be produced via the achievement of the detailed project milestones, constituent tasks and resultant deliverables. The aforementioned are identified, outlined and explained in the chapter titled \textit{Work Plan and Deliverables}.

\newpage
\section{Added Value}
\label{sec:sec01}

It is envisioned that this project will contribute added value by:
\begin{itemize}
    \item Providing the CS faculty with a strategically advantageous option, to deliver curriculum elements via an alternative remote delivery method to reduce the reliance on traditional teaching resources and benefit from both administrative and learning advantages. E.g. Delegating mundane administrative elements or technically simple elements of the curriculum (a common theme in CS courses e.g. specific software to download, recommended IDE's and package managers, development environment options, etc) to the TEL tool and dedicating more time to complex aspects of the curriculum or even allocating more time to practical lab sessions.
    
    \item Remote delivery via the TEL tool may help to eliminate the need for a traditional W-0 and ultimately avoid scheduling clashes with IW, by delivering the curriculum either before IW starts or alternatively during IW but at a  convenient timing subject to IW schedules and student personal preferences.
    
    \item Improve the efficacy of learning objectives by giving students flexible and convenient access to limited course or instructor content well before the start of the course. Students will be able to effectively gauge their development needs and prepare in advance as necessary, benefiting their personal development objectives.
    
    \item Provide a possible 'self-sift/taster' tool for the department's programmes so that students can assess whether the course is technically a good fit for them i.e. Students can consider in advance whether the course is indicative of desirable content and good teaching resources. This may benefit students by saving them resources and the departments administration by providing them with a possible sifting tool.
    
    \item May create a valuable HR resource by allowing staff to retrospectively assess their instruction delivery and develop junior staff (teaching assistants) by delegating TEL curriculum development and activities to them.
    
    \item May help create richer and niche course content by encouraging flexible input towards the curriculum by allowing students to share their own helpful curriculum resources, effectively improving course delivery over time.
    
    \item As the tool will be most likely Open Source in nature, it has the potential to create added value over time. It is envisioned that this may be enabled by allowing future UoB students to improve their TEL solution as per there own preferences or liking.
    
\end{itemize}








%==New Section

\section{Scope and Review Outline}
\label{sec:sec01}

Given the predominant Type I nature of this project most of the work is expected to be on implementation.  However it is hoped that the literature reviewed and resultant research can form a strong theoretical foundation for inspiring the creative and technical aspects for the development of a TEL solution. In light of this most of the research is likely to explore the underlying rationale for developing a TEL solution, inspiring it's creative design and identifying best practice from a technical development perspective.

The following scope is proposed, which effectively determines the review outline:

\begin{itemize}

\item\textbf{TEL \& Pedagogy}: There are a number of possible TEL modalities that can be adopted as a solution, but the  overarching objective of any TEL solution is to be enable a pedagogy that improves resultant learning outcomes.

Therefore \textbf{Chapter 02} will impress upon the relationship between pedagogies, learning methods and effective TEL solutions. The creative design of the TEL solution will be inspired and a modality compatible with findings on the aforementioned for TEL will be selected based on aforementioned rationale; enabling appropriate interaction, a desirable pedagogy and learning outcomes. Open source considerations for TEL solutions will also be considered here.


\item\textbf {Captology, Web Application Security and User Accessibility}: Designing a web application is a very broad topic. This research will focus mostly on captology ("Persuasive Technology Design"), best practice for enhancing web credibility and best practice for web application security. It will conclude with technological considerations and tools to make the TEL solution accessible to users with visual impairments. 

Effectively \textbf{Chapter 03} will cover these areas; creating  a report on best practice for the technical technological consideration during the actual development process of the tool, for realising the creative criteria underpinning decided upon in Chapter 02.

\end{itemize}

\textbf{Chapter 04} will explain a project milestone based approach, with constituent task and resultant deliverables. Adherence to this during the course of the project will hopefully result in a successful outcome.

Finally conclusions on the aforementioned chapters are drawn in \textbf{Chapter 05}, effectively amalgamating their findings.


%=========================================================