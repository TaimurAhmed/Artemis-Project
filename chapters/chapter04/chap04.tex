%
% File: chap04.tex
% Author: ta16969
%
\let\textcircled=\pgftextcircled
\chapter{Workplan and Deliverables}
\label{chap:Workplan and Deliverables}


\initial{T}this chapter will explain the work plan for this project. It will begin by logically listing the project milestones (P) and constituent task (t) required to complete them. Resultant deliverables (d) will be listed as a result of completing tasks or project milestones. This will be illustrated via a Gantt Chart and explained by a work plan. Finally a risk assessment and mitigation strategy will be outlined to conclude this chapter.

\section{Deliverable Summary}
The deliverables summary is illustrated in figure A2 of the appendix. Note that the summary is segregated according to project milestones (P), the constituent tasks (t) of project milestones and resultant deliverables (d) arising towards achieving the tasks that complete the relevant milestone.


\section{Gantt Chart}

Derived from the deliverables list a simple Gantt Chart is appendix in A3 which illustrates the achievement of the deliverable summary. This is explained by the work plan via the achievement of the project milestones.

\section{Risk Assessment and Mitigation}
A risk assessment strategy has been prepared within the scope of a risk matrix,appendixed in A1, that identifies and ranks risks according to likelihood and severity. Appropriate mitigation strategies for identified risks have been outlined in the matrix.

\newpage

\section{Work Plan for Achieving Project Milestones}

The workplan broken up according to project milestones below, flows from the Deliverables Summary and Gannt Chart that are appendixed in A2 and A3 respectively. It explains the how the achievement of the  constituent tasks and resultant deliverables, will lead to the completion of the project.

\subsection{P1 - Planning and Conception}

At this stage the focus is going to be on investigating and producing two reports, along with their constituent sections, to be incorporated into the dissertation at a later date, during P3 and the conclusion of the project.

The first report shall outline the creative design considerations for a TEL solution. The creative design will be inspired by and follow on from the research  on pedagogical,learning and interaction rationale. This design shall be tempered against the result of a brief stakeholder analysis of the students, instructors and the UoB TEL team.

The second report shall outline the technical design considerations for a TEL solution. The technical design will be justified via further investigation into WAI and OWASP best practices. It will have two constituent reports segregated into accessibility and security considerations.

These two reports i.e. technical and creative design consideration, will form the foundation of the development process.

\subsection{P2 - Development and Implementation (Minimum Viable Product)}

At this stage the focus will be on implementing the design considerations into an actual TEL solution. This should result in two deliverables the TEL solution itself and a report explaining the software development strategy and implementation process.

Development is a creative venture and often never a straight forward process. This project has extensive scope and size and the non-optional milestones are reasonably challenging within the constraints of this project. However assuming the best case scenario where targets have been achieved ahead of schedule, optional project extensions outlined under P4 will be considered subject to the constraints.

However should an unfortunate "worst case scenario" materialise during the duration of this project, a risk analysis and mitigation strategy have been outlined in the next chapter. An optional extension would likely not be attempted in such an instance and creating a minimum viable product will be the over riding objective.

\subsection{P3 - Documentation and Dissemination}

At the very least the minimum viable result of this project is a dissertation. Therefore the dissertation document will be priority at this stage. Once the document is completed the next priority will be the development of a poster.

Finally should there be no negative risks materialising at this stage, user documentation via a Wiki page will be produced and developer documentation via a GitHub page will be created. It is believed that this step may improve the usage of the project and longevity as an open source project for future cohorts.

\subsection{P4 - Optional Extensions}
An optional extension may be implemented if the development process end earlier than anticipated. It is proposed that an evaluation strategy e.g. evaluating accessibility, be created via the work resulting from development based on the previous reports regarding design considerations.

Ideally the tool should satisfy these considerations. Alternatively if it does not either improvements to the tool will be considered or alternatively relevant notes will be added to user and developer documentation.





